\documentclass[12pt, a4paper]{article} % Typ dokumentu i rozmiar czcionki
\usepackage[utf8]{inputenc} % Kodowanie znaków (UTF-8)
\usepackage[T1]{fontenc} % Kodowanie fontów
\usepackage[polish]{babel} % Język polski (dzielenie wyrazów, poprawne formatowanie)
\usepackage{graphicx} % Do wstawiania obrazów
\usepackage{amsmath} % Do równań matematycznych
\usepackage{hyperref} % Do klikalnych linków
\usepackage{cite} % Do zarządzania cytowaniami
\usepackage{geometry}
\geometry{a4paper, margin=2.5cm} % Ustawienie marginesów
\usepackage[backend=biber,style=numeric]{biblatex}
\addbibresource{TraficFlowRL.bib} % Nazwa pliku .bib



\title
{Podsumowanie prac nad zastosowaniem Reinforcement Learning do sygnalizacją świetlną w celu optymalizacji
ruchu z wkorzystaniem symulacji SUMO}
\author{Imię i Nazwisko}
\date{\today}

\begin{document}

    \maketitle % Generuje tytuł, autora i datę

    \clearpage
    \tableofcontents
    \clearpage
    \section*{Wprowadzenie}



    Celem niniejszego dokumentu jest przedstawienie dokonań związanych z opracowaniem modelu optymalizującego
    zarządzanie sygnalizacją świetlną w Rzeszowie przy użyciu metod uczenia się ze wzmocnieniem. Projekt rozpoczął się z
    ambitnym założeniem, by objąć symulacją całą mapę miasta, co miało na celu kompleksowe odwzorowanie rzeczywistości i
    uwzględnienie wielu zmiennych wpływających na ruch drogowy. Pomimo rozważania uproszczonych wariantów, na początku
    dominowało podejście obejmujące całość miasta.

    Jednakże po głębszej analizie problemu, przestudiowaniu literatury naukowej oraz ocenie złożoności i
    wielowymiarowości zagadnienia, zdecydowano się na ograniczenie skali badania do pojedynczych skrzyżowań. Takie
    podejście pozwoliło skupić się na przetestowaniu podstawowych mechanizmów i algorytmów w prostszym środowisku, co
    umożliwiło weryfikację założeń przed ewentualnym rozszerzeniem projektu na większą skalę.

    W dotychczasowych pracach przeprowadzono wstępne badania, obejmujące przegląd literatury oraz narzędzi do symulacji
    i uczenia się ze wzmocnieniem. Ponadto wykonano pierwsze testy na uproszczonym scenariuszu symulacyjnym, co
    pozwoliło zidentyfikować podstawowe wyzwania i ograniczenia metod zastosowanych w projekcie. Dokument ten prezentuje
    wyniki tych prac oraz wnioski, które mogą stanowić podstawę do dalszego rozwoju projektu.


    \section{Złożoność sterowania ruchem miejskim za pomocą sygnalizacji świetlnej}

    Sterowanie ruchem miejskim jest zadaniem niezwykle złożonym, co wynika z wielu czynników wpływających na dynamikę
    ruchu w obszarach miejskich. Nawet przy wykorzystaniu zaawansowanych narzędzi symulacyjnych, takich jak
    \textit{Simulation of Urban MObility} (SUMO)
    , oraz przy założeniu eliminacji czynników losowych, problem ten pozostaje wielowymiarowy i skomplikowany.

    \subsection{Wielowymiarowość problemu}

    Wielowymiarowość sterowania ruchem miejskim odnosi się do liczby i różnorodności zmiennych, które należy uwzględnić
    w procesie decyzyjnym. Do kluczowych aspektów należą:

    \begin{itemize}
        \item \textbf{Struktura sieci drogowej}
        : Złożoność układu dróg, skrzyżowań, rond i innych elementów infrastruktury.
        \item \textbf{Rodzaje uczestników ruchu}
        : Obecność różnych typów pojazdów (samochody osobowe, autobusy, rowery) oraz pieszych.
        \item \textbf{Zmienność czasowa}: Zmiany natężenia ruchu w zależności od pory dnia, dnia tygodnia czy sezonu.
        \item \textbf{Interakcje między uczestnikami ruchu}
        : Zachowania kierowców, pieszych i innych uczestników ruchu wpływające na płynność i bezpieczeństwo.
        \item \textbf{Polityki zarządzania ruchem}
        : Zasady pierwszeństwa, sygnalizacja świetlna, ograniczenia prędkości i inne regulacje.
    \end{itemize}

    Każdy z tych wymiarów wprowadza dodatkową warstwę złożoności, co sprawia, że modelowanie i optymalizacja ruchu
    miejskiego wymagają uwzględnienia wielu czynników jednocześnie.

%    \subsection{Deterministyczne modelowanie a złożoność}
%
%    Nawet po wyeliminowaniu czynników losowych i przyjęciu założeń deterministycznych, sterowanie ruchem miejskim
%    pozostaje skomplikowane. Deterministyczne modele zakładają, że przy tych samych warunkach początkowych system
%    zachowa się w identyczny sposób. Jednak w praktyce:
%
%    \begin{itemize}
%        \item \textbf{Nieliniowość systemu}
%        : Relacje między zmiennymi nie są liniowe, co utrudnia przewidywanie skutków wprowadzanych zmian.
%        \item \textbf{Sprzężenia zwrotne}
%        : Działania podjęte w jednym obszarze mogą wpływać na inne części systemu, prowadząc do nieoczekiwanych
%        konsekwencji.
%        \item \textbf{Ograniczenia obliczeniowe}
%        : Dokładne modelowanie wszystkich aspektów ruchu miejskiego wymaga ogromnej mocy obliczeniowej i zaawansowanych
%        algorytmów.
%    \end{itemize}

    \section{Przegląd literatury dotyczącej współczesnych rozwiązań i narzędzi w zakresie sterowania ruchem sygnalizacji
    świetlnej oraz uczenia modeli z wykorzystaniem metod uczenia ze wzmocnieniem}

    \subsection{Czym jest Simulation of Urban MObility (SUMO) dlaczego został wybrany?}

    \textit{Simulation of Urban MObility} (SUMO)
    to otwartoźródłowy, przenośny, mikroskopowy i ciągły symulator ruchu drogowego, zaprojektowany do obsługi dużych
    sieci. Rozwijany przez Niemieckie Centrum Lotnictwa i Kosmonautyki (DLR) od 2001 roku, SUMO umożliwia modelowanie
    systemów transportu intermodalnego, w tym pojazdów drogowych, transportu publicznego oraz pieszych. Dzięki bogatemu
    zestawowi narzędzi wspierających, takich jak wyszukiwanie tras, wizualizacja, import sieci czy kalkulacja emisji,
    SUMO stał się wszechstronnym narzędziem do analizy ruchu miejskiego \cite{sumo_website}.

    \subsection{Dlaczego SUMO zostało wybrane jako środowisko symulacyjne}

    Uczenie ze wzmocnieniem (RL) wymaga środowisk symulacyjnych, które są zarówno realistyczne, jak i elastyczne. SUMO
    spełnia te kryteria z kilku powodów:

    \begin{itemize}
        \item \textbf{Dokładność mikroskopowa}
        : SUMO symuluje ruch na poziomie pojedynczych pojazdów i pieszych, co pozwala na precyzyjne modelowanie
        interakcji w ruchu drogowym \cite{sumo_docs}.
        \item \textbf{Wsparcie dla różnych trybów transportu}
        : Możliwość modelowania różnych środków transportu, takich jak samochody, autobusy, rowery czy piesi, umożliwia
        tworzenie złożonych scenariuszy miejskich.
        \item \textbf{Interfejsy API do zdalnego sterowania}
        : SUMO oferuje interfejsy API, takie jak TraCI, które pozwalają na zdalne sterowanie symulacją w czasie
        rzeczywistym, co jest kluczowe dla implementacji algorytmów RL \cite{sumo_docs}.
        \item \textbf{Integracja z bibliotekami RL}
        : Istnieją narzędzia, takie jak SUMO-RL, które upraszczają integrację SUMO z popularnymi bibliotekami RL,
        umożliwiając szybkie tworzenie i testowanie środowisk dla kontroli sygnalizacji świetlnej \cite{sumo_rl}.
        \item \textbf{Wszechstronność i rozwój SUMO}
        : SUMO jest nieocenionym narzędziem w badaniach nad ruchem drogowym, oferującym zaawansowane funkcje w zakresie
        rozwiązań intermodalnych, łączenia symulatorów oraz rozwoju i walidacji modeli. Jego otwartość i ciągły rozwój
        czynią go kluczowym narzędziem w symulacjach mikroskopowych ruchu drogowego \cite{Lopez2018}.
    \end{itemize}

    Dzięki tym cechom, SUMO jest idealnym narzędziem do badań nad zastosowaniem uczenia ze wzmocnieniem w zarządzaniu
    ruchem miejskim, pozwalając na tworzenie realistycznych i kontrolowanych środowisk symulacyjnych.

    \subsubsection{Wykorzystanie SUMO jako środowiska w uczeniu ze wzmocnieniem}

    SUMO jest zaawansowanym narzędziem symulacyjnym, które znajduje szerokie zastosowanie
    w badaniach nad uczeniem ze wzmocnieniem (Reinforcement Learning, RL). Dzięki swoim funkcjonalnościom SUMO umożliwia
    modelowanie złożonych scenariuszy ruchu drogowego, wspierając procesy uczenia agentów w dynamicznie zmieniających
    się środowiskach.

    Najważniejsze cechy wykorzystania SUMO w kontekście uczenia ze wzmocnieniem obejmują:

    \begin{itemize}
        \item \textbf{Tworzenie wielu agentów}:
        SUMO pozwala na definiowanie różnych agentów (np. sygnalizacji świetlnej, pojazdów, pieszych), z których każdy
        może podejmować swoje unikalne akcje w odpowiedzi na zmieniające się warunki ruchu.
        \item \textbf{Pozyskiwanie szczegółowych informacji z symulacji}:
        Narzędzie umożliwia dostęp do bogatego zestawu danych, takich jak prędkość pojazdów, natężenie ruchu, czas
        oczekiwania na skrzyżowaniach czy zagęszczenie ruchu. Te dane są kluczowe do definiowania stanów środowiska,
        które są wykorzystywane przez agentów w procesie podejmowania decyzji.

        \item \textbf{Wyznaczanie nagród}:
        SUMO pozwala na projektowanie mechanizmów nagradzania w oparciu o konkretne cele, takie jak minimalizacja czasu
        podróży, redukcja korków, czy zwiększenie bezpieczeństwa uproszczony schemat działania nagrody oraz obserwacji \ref{fig:sumo_env}. Dzięki temu agenci uczą się podejmować optymalne
        decyzje w oparciu o dynamiczne zmiany w symulowanym środowisku.
        \begin{figure}[!h]
        \centering
        \includegraphics[width=0.8\textwidth]{fig/sumo_env}
        \caption{Agent RL do sterowania sygnalizacją świetlną. Agent obserwuje aktualny stan ruchu,
            podejmuje decyzje o zmianie sygnalizacji i otrzymuje nagrody na podstawie poprawy przepływu ruchu.}
        \label{fig:sumo_env}
    \end{figure}


        \item \textbf{Elastyczność w modelowaniu scenariuszy}:
        Narzędzie umożliwia symulację różnorodnych sytuacji drogowych, co pozwala testować skuteczność uczenia ze
        wzmocnieniem w warunkach bliskich rzeczywistości.
    \end{itemize}

    SUMO, jako środowisko symulacyjne, dostarcza podstaw do tworzenia zaawansowanych algorytmów RL, które mogą być
    wykorzystywane do optymalizacji systemów sterowania ruchem drogowym, zwiększając płynność ruchu oraz minimalizując
    negatywne skutki zatorów komunikacyjnych.

%
%    \section{Złożoność sterowania sygnalizacją świetlną w miastach z wykorzystaniem Reinforcement Learning}
%
%    Zarządzanie sygnalizacją świetlną w miastach to złożone zadanie, wynikające z dynamicznej i nieprzewidywalnej
%    natury
%    przepływu ruchu. Tradycyjne metody sterowania często okazują się niewystarczające w optymalizacji efektywności
%    ruchu, co skłania badaczy do eksploracji adaptacyjnych podejść, takich jak Reinforcement Learning (RL). Modele
%    RL
%    mogą uczyć się optymalnych strategii sterowania sygnalizacją świetlną, poprzez interakcję ze środowiskiem ruchu
%    drogowym, dążąc do minimalizacji korków i poprawy płynności ruchu.
%
%    \begin{figure}[h]
%        \centering
%        \includegraphics[width=0.8\textwidth]{fig/learning_agent}
%        \caption{Agent RL do sterowania sygnalizacją świetlną. Agent obserwuje aktualny stan ruchu,
%            podejmuje decyzje o zmianie sygnalizacji i otrzymuje nagrody na podstawie poprawy przepływu ruchu.
%            \cite{mousavi2017}}
%        \label{fig:rl_traffic_signal}
%    \end{figure}
%
%    Jak przedstawiono na Rysunku~\ref{fig:rl_traffic_signal}
%    , agent RL obserwuje aktualny stan środowiska ruchu, takie jak pozycje pojazdów i długości kolejek, wybiera
%    działanie polegające na dostosowaniu sygnalizacji świetlnej i otrzymuje nagrodę w zależności od efektywności
%    przepływu ruchu. Ta interakcja odbywa się iteracyjnie, co pozwala agentowi na naukę optymalnych strategii
%    sterowania
%    sygnalizacją świetlną w czasie.
%
%    Mousavi i in. (2017) przedstawili agenta RL opartego na głębokim uczeniu, przeznaczonego do sterowania
%    sygnalizacją
%    świetlną, szczegółowo opisując interakcję agenta z otoczeniem. W ich badaniu zamieszczono diagram ilustrujący
%    elementy stanu, działania i nagrody w ramach modelu RL, co zapewnia jasny obraz procesu podejmowania decyzji
%    przez
%    agenta \cite[Figura~1]{mousavi2017}.
%
%    Wei i Zhang (2022) przeanalizowali zastosowanie głębokiego RL w optymalizacji sygnalizacji świetlnej. Ich
%    badania
%    zawierają diagramy przedstawiające strukturę modelu RL, środowisko symulacyjne oraz różne scenariusze ruchu
%    użyte do
%    trenowania i testowania. Te ilustracje pomagają zrozumieć architekturę modelu i jego interakcję z różnymi
%    warunkami
%    ruchu \cite[Figury~2~i~3]{wei2022}.
%
%    Dodatkowo repozytorium GitHub „Reinforcement Learning for Traffic Signal Control” oferuje obszerną kolekcję
%    zasobów,
%    w tym kluczowe artykuły, otwarte zbiory danych oraz symulatory ruchu. Repozytorium zawiera diagramy i
%    wizualizacje
%    wyjaśniające różne podejścia RL i ich implementacje w systemach sterowania sygnalizacją świetlną, co stanowi
%    cenne
%    odniesienie dla praktycznego wdrażania \cite{rlsignals}.
%
%    Te zasoby wspólnie podkreślają złożoność sterowania sygnalizacją świetlną i pokazują, jak modele RL można
%    strukturyzować i stosować do zarządzania ruchem miejskim.


    \section{Przykładowe badania dotyczące optymalizacji sygnalizacji świetlnej przy użyciu uczenia ze wzmocnieniem}

    \subsection{Cel optymalizacji}
    Algorytmy uczenia ze wzmocnieniem (DQN i PPO) zostały zastosowane do optymalizacji sygnalizacji świetlnej na
    skrzyżowaniu, aby zmaksymalizować płynność ruchu i zminimalizować negatywne skutki związane z korkami. Główne cele
    optymalizacji obejmowały:

    \begin{itemize}
        \item \textbf{Minimalizacja czasu oczekiwania pojazdów}
        – redukcja czasu, przez jaki pojazdy muszą czekać na zielone światło \cite{Louw2022}.
        \item \textbf{Skrócenie długości kolejek} – zapobieganie tworzeniu się zatorów na pasach ruchu \cite{Louw2022}.
        \item \textbf{Poprawa średniej prędkości ruchu}
        – zwiększenie prędkości pojazdów poruszających się przez skrzyżowanie \cite{Louw2022}.
    \end{itemize}

    \subsection{Równanie Webstera w optymalizacji sygnalizacji świetlnej jako tradycyjne podejście do optymalizacji}

    Równanie Webstera było jednym z tradycyjnych podejść do optymalizacji sygnalizacji świetlnej, które zostało
    porównane z metodami uczenia ze wzmocnieniem (RL) w tym badaniu \cite{Webster1958}. Równanie to, opisane w sekcji
    \textbf{2.1 Fixed-time control}
    , służy do obliczania optymalnego czasu cyklu sygnalizacji świetlnej, minimalizując całkowite opóźnienie na
    skrzyżowaniu. Równanie Webstera jest zdefiniowane jako:

    \[
        C_{o} = \frac{1.5 \cdot L + 5}{1 - \sum Y_{i}},
    \]

    gdzie:
    \begin{itemize}
        \item \( C_{o} \) to optymalny czas cyklu w sekundach,
        \item \( L \) to całkowity czas stracony na cykl w sekundach,
        \item \( Y_{i} \) to stosunek natężenia ruchu do przepustowości dla każdego krytycznego ruchu w fazie \( i \).
    \end{itemize}

    Mimo że równanie Webstera jest skuteczne w statycznych warunkach ruchu, nie radzi sobie z dynamicznymi zmianami w
    natężeniu ruchu, co czyni je mniej elastycznym w porównaniu z nowoczesnymi metodami RL, takimi jak DQN i PPO
    \cite{Louw2022}.

    \subsection{Przykład badania}
    Badania przeprowadzono w symulacji z wykorzystaniem oprogramowania SUMO (Simulation of Urban MObility), które
    zostało zintegrowane z OpenAI Gym, co umożliwiło zastosowanie algorytmów RL w środowisku symulacyjnym
    \cite{Louw2022}.

    \subsubsection{Konfiguracja skrzyżowania}
    \begin{itemize}
        \item \textbf{Model skrzyżowania}
        : Proste skrzyżowanie z 12 pasami ruchu (każdy pas monitorowany pod kątem długości kolejek i liczby zatrzymanych
        pojazdów) \cite{Louw2022}.
        \item \textbf{Definicje faz świateł}
        : Zdefiniowano różne fazy, np. umożliwiające ruch w kierunku N-S, E-W lub dedykowane dla skrętów
        \cite{Allsop1971}.
    \end{itemize}

    \begin{figure}[h]
        \centering
        \includegraphics[width=0.4\textwidth]{fig/intersection_layout}
        \caption{Geometria skrzyżowania użytego w badaniach \cite{Louw2022}.}
        \label{fig:intersection}
    \end{figure}

    \subsubsection{Wzorce ruchu}
    Przetestowano cztery różne wzorce ruchu, aby odzwierciedlić rzeczywiste scenariusze:

    \begin{itemize}
        \item \textbf{P1}: Wysokie natężenie ruchu na głównych drogach \cite{Louw2022}.
        \item \textbf{P2}: Większy ruch na pasach do skrętu niż na pasach na wprost \cite{Louw2022}.
        \item \textbf{P3}
        : Kierunkowy ruch przypominający zjawisko „pływów” – duże natężenie na dwóch prostopadłych drogach
        \cite{Louw2022}.
        \item \textbf{P4}: Zmienne natężenie ruchu w czasie na jednej z głównych dróg \cite{Louw2022}.
    \end{itemize}

    \begin{figure}[h]
        \centering
        \includegraphics[width=\textwidth]{fig/traffic_patterns}
        \caption{Wzorce ruchu użyte w badaniach \cite{Louw2022}.}
        \label{fig:traffic_patterns}
    \end{figure}

    \subsection{Nagroda dla agenta RL}
    Nagroda została zaprojektowana w sposób prosty i oparta na karze za czas oczekiwania pojazdów:

    \[
        R_t = -\sum_{i=1}^{n} w_i^t
    \]

    gdzie:
    \begin{itemize}
        \item \( R_t \) – nagroda w czasie \( t \),
        \item \( n \) – liczba pojazdów czekających na światłach,
        \item \( w_i^t \) – czas oczekiwania \( i \)-tego pojazdu \cite{Louw2022}.
    \end{itemize}

    Algorytmy były uczone, aby maksymalizować sumaryczną nagrodę, co prowadziło do redukcji czasu oczekiwania i poprawy
    płynności ruchu \cite{Mnih2015}.

    \subsection{Porównanie wyników}
    DQN i PPO, czyli metody uczenia ze wzmocnieniem (RL), znacznie poprawiły wyniki w porównaniu z tradycyjnymi
    metodami, takimi jak równanie Webstera, redukując maksymalny czas oczekiwania do około 50 sekund i zapewniając
    bardziej równomierny przepływ ruchu \cite{Louw2022}
    . Oba algorytmy były skuteczne w radzeniu sobie z dynamicznymi zmianami we wzorcach ruchu, co czyni je lepszym
    wyborem do zastosowań w rzeczywistych systemach \cite{Louw2022}.

    W przeciwieństwie do równania Webstera, które jest skuteczne tylko w statycznych warunkach ruchu, metody RL (DQN i
    PPO) wykazały zdolność do adaptacji do zmiennych warunków ruchu, co przekłada się na lepszą efektywność w
    zarządzaniu sygnalizacją świetlną. Jak pokazano na rysunkach \ref{fig:phase_webster}, \ref{fig:phase_dqn} i
    \ref{fig:phase_ppo}
    , metody RL są znacznie bardziej elastyczne w przełączaniu faz świateł, dostosowując się do aktualnych potrzeb
    ruchu. W przypadku metody Webstera sekwencje faz są stałe i niezmienne, podczas gdy DQN i PPO dynamicznie zmieniają
    kolejność faz, aby zminimalizować opóźnienia i długość kolejek.

    \begin{figure}[h]
        \centering
        \includegraphics[width=\textwidth]{fig/waiting_time_comparison}
        \caption{Porównanie czasu oczekiwania dla różnych metod \cite{Louw2022}.}
        \label{fig:waiting_time}
    \end{figure}

    \subsection{Sekwencje faz świateł}
    Poniżej przedstawiono przykładowe sekwencje faz świateł dla różnych metod:

    \begin{figure}[h]
        \centering
        \includegraphics[width=0.6\textwidth]{fig/phase_sequence_webster}
        \caption
        {Sekwencja faz świetlnych dla metody Webstera \cite{Louw2022}. Metoda ta opiera się na stałych sekwencjach,
            które nie dostosowują się do zmiennych warunków ruchu.}
        \label{fig:phase_webster}
    \end{figure}

    \begin{figure}[h]
        \centering
        \includegraphics[width=0.8\textwidth]{fig/phase_sequence_dqn}
        \caption
        {Sekwencja faz świetlnych dla metody DQN \cite{Louw2022}. Algorytm DQN dynamicznie dostosowuje kolejność faz,
            aby zoptymalizować przepływ ruchu.}
        \label{fig:phase_dqn}
    \end{figure}

    \begin{figure}[!h]
        \centering
        \includegraphics[width=0.8\textwidth]{fig/phase_sequence_ppo.png}
        \caption{Sekwencja faz świetlnych dla metody PPO \cite{Louw2022}. Podobnie jak DQN,
            PPO wykazuje elastyczność w przełączaniu faz, ale z większym naciskiem na stabilność polityk.}
        \label{fig:phase_ppo}
    \end{figure}

    \subsection{Wnioski}
    Badania wykazały, że algorytmy uczenia ze wzmocnieniem (DQN i PPO) są skuteczne w optymalizacji sygnalizacji
    świetlnej, zapewniając lepszą płynność ruchu i redukując czas oczekiwania pojazdów w porównaniu z tradycyjnymi
    metodami \cite{Louw2022}. Jak pokazano na rysunkach \ref{fig:phase_dqn} i \ref{fig:phase_ppo}
    , metody RL są znacznie bardziej elastyczne w przełączaniu faz świateł, dostosowując się do aktualnych potrzeb
    ruchu. W przeciwieństwie do metody Webstera (patrz rysunek \ref{fig:phase_webster}
    ), która opiera się na stałych sekwencjach, DQN i PPO dynamicznie zmieniają kolejność faz, aby zminimalizować
    opóźnienia i długość kolejek.

    Wyniki sugerują, że metody te mogą być z powodzeniem zastosowane w rzeczywistych systemach zarządzania ruchem
    drogowym, szczególnie w środowiskach o zmiennym natężeniu ruchu. Elastyczność i zdolność do adaptacji metod RL
    sprawiają, że są one lepszym rozwiązaniem w porównaniu do tradycyjnych, statycznych metod sterowania sygnalizacją
    świetlną \cite{Louw2022}.


    \section{Przygotowanie danych do symulacji i implementacji}

    Jak wspomniano we wstępie, na początku projektu przyjęto ambitne założenia, aby trenować modele na pełnej mapie
    miasta Rzeszów. W związku z tym opracowano i przygotowano mapy Rzeszowa w trzech różnych wariantach dokładności,
    które mogą być wykorzystane w symulacjach. Poniżej przedstawiono obrazy tych wariantów:

    \begin{itemize}
        \item \textbf{Wariant 1}
        : Mapy o niskiej szczegółowości, optymalizowane pod kątem szybkich symulacji, mapa pokazana na rysunku
        \ref{fig:mapa_wariant1}.
        \item \textbf{Wariant 2}
        : Mapy o średniej szczegółowości, równoważące szczegółowość i wydajność, mapa pokazana na rysunku
        \ref{fig:mapa_wariant2}.
        \item \textbf{Wariant 3}
        : Mapy o wysokiej szczegółowości, zawierające precyzyjne odwzorowanie infrastruktury miejskiej co warto zauważyć
        to nie jest maksymalna dokłdaność mapy tylko tak do której chcieliśmy doążyć, mapa pokazana na
        rysunku \ref{fig:mapa_wariant3}.
    \end{itemize}

    \begin{figure}[!h]
        \centering
        \includegraphics[width=0.5\textwidth]{fig/s}
        \caption{Mapa Rzeszowa w wariancie 1: niska szczegółowość, zoptymalizowana pod kątem szybkich symulacji.}
        \label{fig:mapa_wariant1}
    \end{figure}

    \begin{figure}[!h]
        \centering
        \includegraphics[width=0.5\textwidth]{fig/m}
        \caption{Mapa Rzeszowa w wariancie 2: średnia szczegółowość, równoważąca szczegółowość i wydajność.}
        \label{fig:mapa_wariant2}
    \end{figure}

    \begin{figure}[!h]
        \centering
        \includegraphics[width=0.5\textwidth]{fig/l}
        \caption
        {Mapa Rzeszowa w wariancie 3: wysoka szczegółowość, z precyzyjnym odwzorowaniem infrastruktury miejskiej.}
        \label{fig:mapa_wariant3}
    \end{figure}

    Dane do symulacji zostały pobrane z platformy OpenStreetMap (OSM) dla obszaru Rzeszowa. W celu przystosowania ich do
    użycia w symulatorze SUMO, dane te zostały odpowiednio zmniejszone i przekonwertowane do formatów obsługiwanych
    przez SUMO, takich jak \texttt{.net.xml} (sieć drogowa), \texttt{.rou.xml} (trasy pojazdów) oraz
    \texttt{.add.xml} (dodatkowe elementy infrastruktury)
    . Do przetwarzania danych wykorzystano następujące narzędzia dostarczane przez SUMO:

    \begin{itemize}
        \item \textbf{\texttt{netconvert}}: Konwersja danych OSM do formatu \texttt{.net.xml}
        , umożliwiającego reprezentację sieci drogowej w SUMO.
        \item \textbf{\texttt{polyconvert}}
        : Przetwarzanie danych OSM w celu wygenerowania informacji o obszarach (np. budynkach) do wykorzystania w
        wizualizacjach symulacji.
        \item \textbf{\texttt{randomTrips.py}}: Generowanie losowych tras pojazdów w formacie \texttt{.rou.xml}
        na podstawie zdefiniowanej sieci drogowej.
    \end{itemize}

    W ramach prac nad symulacjami opracowano również algorytm do losowego wyznaczania tras pojazdów na całej mapie
    Rzeszowa. Algorytm ten wykorzystano do stworzenia punktu odniesienia, który pozwolił na analizę zachowania systemu w
    kontekście pełnej sieci drogowej.

    \begin{figure}[!h]
        \centering
        \includegraphics[width=0.8\textwidth]{fig/BS}
        \caption
        {Przykładowe skrzyżowanie z ruchem w symulacji SUMO. Widoczne są pojazdy poruszające się w różnych kierunkach
        oraz sygnalizacja świetlna.}
        \label{fig:skrzyzowanie_ruch}
    \end{figure}

    Pomimo wstępnych założeń i zaawansowanego przygotowania, po głębszej analizie problemu oraz zrozumieniu jego
    złożoności, podjęto decyzję o zmianie podejścia. Zdecydowano się uprościć początkowe założenia i skupić na
    stworzeniu funkcjonalnego i intuicyjnego interfejsu


    \section{Odczyt danych z symulacji}

    W naszym eksperymencie dane z symulacji są odczytywane za pomocą narzędzia
    \textbf{TraCI} (Traffic Control Interface)
    , które umożliwia komunikację z symulatorem SUMO w czasie rzeczywistym. Dzięki TraCI możemy zbierać informacje o
    ruchu drogowym, takie jak:
    \begin{itemize}
        \item Liczba pojazdów na poszczególnych pasach ruchu.
        \item Długość kolejek.
        \item Prędkość pojazdów.
        \item Czas oczekiwania na światłach.
    \end{itemize}

    Te dane są następnie przetwarzane w celu zbudowania \textbf{mapy natężenia ruchu}
    , która reprezentuje aktywność ruchową na danym obszarze. Mapa ta jest wizualizowana na rysunku
    \ref{fig:mapa_natezenia_ruchu} i stanowi kluczowy element w naszych badaniach.

    \begin{figure}[h]
        \centering
        \includegraphics[width=0.8\textwidth]{fig/mapa_natezenia_ruchu}
        \caption
        {Mapa natężenia ruchu zbudowana na podstawie danych z symulacji SUMO. Kolory reprezentują natężenie ruchu: od
        niskiego (zielony) do wysokiego (czerwony).}
        \label{fig:mapa_natezenia_ruchu}
    \end{figure}

    \subsection{Budowa mapy natężenia ruchu}
    Mapa natężenia ruchu jest generowana na podstawie danych odczytywanych z symulacji za pomocą TraCI. Proces ten można
    podzielić na następujące kroki:

    \begin{enumera<te}
        \item \textbf{Zbieranie danych}:
        \begin{itemize}
            \item Dla każdego pasa ruchu odczytywana jest liczba pojazdów (
            \texttt{traci.lane.getLastStepVehicleNumber(lane\_id)}) oraz długość kolejki (
            \texttt{traci.lane.getLastStepHaltingNumber(lane\_id)}).
            \item Dodatkowo zbierane są informacje o prędkości pojazdów (\texttt{traci.vehicle.getSpeed(vehicle\_id)}
            ) i czasie oczekiwania (\texttt{traci.lane.getWaitingTime(lane\_id)}).
        \end{itemize}

        \item \textbf{Agregacja danych}:
        \begin{itemize}
            \itemDane z poszczególnych pasów ruchu są agregowane w celu uzyskania informacji o natężeniu ruchu na całym
            obszarze.
            \item Każdy obszar (np. fragment mapy) jest reprezentowany jako piksel na mapie natężenia ruchu.
        \end{itemize}

        \item \textbf{Normalizacja i wizualizacja}:
        \begin{itemize}
            \item
            Zebrane dane są normalizowane, aby można je było przedstawić na skali od 0 do 1, gdzie 0 oznacza brak ruchu,
            a 1 — maksymalne natężenie.
            \item
            Następnie dane są wizualizowane jako mapa kolorów, gdzie kolory reprezentują różne poziomy natężenia ruchu
            (np. zielony — niskie, żółty — średnie, czerwony — wysokie).
        \end{itemize}
    \end{enumerate}

    \subsection{Wykorzystanie mapy natężenia ruchu w sieciach konwolucyjnych}
    Mapa natężenia ruchu jest kluczowym elementem w naszych badaniach, ponieważ pozwala na wykorzystanie
    \textbf{sieci konwolucyjnych (CNN)}
    do analizy ruchu drogowego. Dzięki takiej reprezentacji sieć może uczyć się nie tylko wartości liczbowych (np.
    liczby pojazdów), ale także \textbf{przestrzennych zależności}
    między różnymi obszarami mapy. Oto, co sieć może uwzględniać:
    \begin{itemize}
        \item \textbf{Umiejscowienie natężenia ruchu}: Sieć może rozpoznawać, gdzie występują korki lub wolne obszary.
        \item \textbf{Kierunek ruchu}
        : Dzięki analizie zmian natężenia w czasie sieć może przewidywać, w którą stronę porusza się ruch.
        \item \textbf{Wzorce ruchu}: Sieć może identyfikować powtarzające się wzorce ruchu, np. godziny szczytu.
    \end{itemize}

    Takie podejście pozwala na bardziej zaawansowaną analizę i optymalizację ruchu drogowego, ponieważ sieć konwolucyjna
    może uwzględniać nie tylko lokalne, ale również globalne zależności w sieci drogowej.

    \subsection{Przykład zastosowania}
    W naszym eksperymencie mapa natężenia ruchu jest wykorzystywana do trenowania modelu uczenia maszynowego, który ma
    na celu optymalizację sygnalizacji świetlnej. Model analizuje mapę natężenia ruchu i na jej podstawie podejmuje
    decyzje o zmianie faz świateł, aby zminimalizować czas oczekiwania i długość kolejek.

    \subsection{Podsumowanie}
    Dzięki wykorzystaniu narzędzia TraCI oraz budowie mapy natężenia ruchu możemy efektywnie analizować i optymalizować
    ruch drogowy w symulacjach. Mapy te są kluczowe dla zastosowania sieci konwolucyjnych, które mogą uwzględniać
    zarówno wartości liczbowe, jak i przestrzenne zależności w ruchu drogowym. W przyszłości planujemy rozszerzyć to
    podejście na większe obszary miejskie, aby jeszcze bardziej poprawić efektywność systemów sterowania ruchem.



    \clearpage
    \printbibliography
    %bitehack 2022-3  1. miejsce AI
\end{document}